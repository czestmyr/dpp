\documentclass[a4paper,10pt]{article}
\usepackage[utf8x]{inputenc}
\usepackage{a4wide}

%opening
\title{Zdůvodnění návrhu k 2.úkolu na DPP}
\author{Martin Bulant a Čestmír Houška}

\begin{document}

\maketitle

\section*{Zvolené technologie}

Pro naše řešení knihovny jsme si zvolili jazyk C++, se kterým máme oba největší zkušenosti. Po prozkoumání
požadavků ze zadání jsme se rozhodli, že až na občasné počítání referencí a parsování řetězců nebude potřeba
používat žádnou jinou knihovnu, než standartní STL. Proto jsme si počítání referencí a parsování napsali také
sami, abychom snížili závislost našeho řešení na jiných knihovnách, což je u tak jednoduché knihovny, jako
je ta naše, žádoucí.

\section*{Základní struktura tříd}

Knihovna se skládá z několika hlavních tříd, které se starají o jednotlivé podúkoly při parsování parametrů.
Těmi podúkoly jsou naparsování vstupních stringů a jejich rozdělení na jednotlivé tokeny -- o to se stará třída \verb|ArgumentParser|.
Dále je třeba uchovat naparsovaná data -- tento úkol řeší třída \verb|ArgumentData|.
Dříve než je ovšem možné parsovat, je třeba nastavit a uložit syntaxi, kterou daný program vyžaduje, to zajišťuje třída \verb|OptionSyntax|.

Pro jednoduchost parsovací úlohy jsme se rozhodli pro parserem řízené zpracování. To znamená, že hlavní slovo
má třída \verb|ArgumentParser|, která při parsování vstupu využívá informací ze třídy \verb|OptionSyntax| a \verb|ArgumentData|.
Vzhledem k tomu, že se zde komunikuje pouze se dvěma třídami, jsou tyto odkazovány přímo z \verb|ArgumentParseru|
a není použit návrhový vzor Mediator.

Pro zjednodušní práce s knihovnou je celá tato struktura zastřešena třídou \verb|FrontEnd|, která poskytuje metody
potřebné pro práci s knihovnou. Veškeré operace, jako definování parametrů či zjišťování jestli byl
daný parametr nastaven, se provádí skrz ni. Tuto třídu lze i kopírovat, čímž je možné si uchovat stav knihovny
spolu s uloženými argumenty.

Při parsování je dále využívána třída \verb|StringParser|, která obsahuje pomocné funkce pro parsovaní řetězců. Tato třída
vlastně zaobaluje postupné porovnávání jednotlivých znaků v řetězci a posouvání ukazatele na znak v něm.

\section*{Rozhraní voleb}

Definování voleb, jejich parametrů, či případných synonym se provádí srz volání metod na třídě \verb|FrontEnd|. Tento způsob
byl zvolen díky možnosti okamžité kontroly zadávaných hodnot vůči již definovaným volbám. Což by při použití
zastřešující struktury pro volbu a nastavování parametrů přímo na ní nebylo možné. Druhým důvodem bylo, že nám tento
způsob přišel přehlednější. 

Reakci na předání volby ve formě callbacků jsme zavrhli, jelikož parsování argumentů probíhá většinou spíše na začátku
programu. Přičemž callbacky by byly vhodnější pro volání v celém průběhu programu. Zde by se pomocí nich stejně
pouze nastavovaly jen hodnoty nějakých proměn-ných. Je podle nás tedy mnohem lepší tyto hodnoty rovnou uložit někde
uvnitř knihovny a nechat uživatele, aby si je vytáhl podle potřeby.

Získání voleb v jednom velkém seznamu nám také nepřišlo vhodné, neboť by si uživatel pak stejně musel zjišťovat, které
volby byly zadány. Proto jsme mu práci zjednodušili a umožnili ptát se na to, jestli volba byla definována, přímo naší knihovny.

\section*{Typy a hodnoty parametrů}

Pro definování typu parametru jsme využili dědění od základního typu \verb|Type|, který definuje potřebné operace,
využívané naší knihovnou. Tento způsob jsme zvolili z důvodu velmi jednodu-chého rozšiřování knihovny o nové typy
možných argumentů pro pokročilejší uživatele. Každý typ pak rozhoduje v metodě \verb|fromString|, jak bude parsovat
vstupní řetězce a kdy vyhodí jakou chybu.

Polymorfismus hodnot parametrů voleb jsme vyřešili opět děděním -- tentokrát od třídy \verb|Value|. Důvodem pro dvě
hierarchie je možnost mít více typů, které budou vracet stejnou hodnotu (například soubor a výčtový typ budou oba
vracet řetězcovou hodnotu, přestože se jedná o různé typy).

Získávání hodnot je řešeno pomocí šablonové metody, která provede dynamické přetypování z třídy \verb|Value| na jejího příslušného
potomka. To, o jakého potomka se jedná, se zjistí z typu hodnoty, kterou má knihovna vrátit. Takže pokud například
uživatel napíše:
\begin{verbatim}arglib.getOptionParameter<int>("volba")\end{verbatim}
bude se uložená hodnota přetypovávat na \verb|IntValue|. Tento
postup je sice poněkud náchylný na chyby (když uživatel zadá nesprávný šablonový typ), ale tyto chyby lze díky výjimkám
snadno odhalit a výsledný zdrojový uživatelský kód nám takto přišel nejelegantnější.

\section*{Chybové stavy}

Kontrolu voleb parametrů a jejich typů provádí knihovna implicitně. Pokud parametry předané programu nebudou
korektní, nemá cenu v práci s programem dále pokračovat.

Pro ohlašování chyb jsme použili výjimky, jelikož umožňují používat jednodušší rozhraní funkcí, které provádějí
jednotlivé operace na vnitřních třídách. V opačném případě by bylo nutné pro funkce vracející hodnotou vracet
informaci o úspěchu pomocí reference v parametrech či kódováním do návratové hodnoty. Další výhodou pak bylo
velké zjednodušení konstrukcí pro řízení toku uvnitř knihovny.

Použili jsme vlastní typ výjimky s odpovídajícím názvem pro práci při zpracování agrumentů.
Vzhledem k malému rozsahu jsme definovali pouze jeden typ a přesnější určení provádíme v rámci textového popisu,
který je součástí výjimky. Vzhledem k tomu, že nepředpokládáme žádné automatické zpracování výjimky, nanejvýš
vypsání její chybové hlášky a skončení programu, je pouhý textový popis naprosto dostačující.

Hlášení uživateli programu knihovna přímo nevypisuje, všechny chyby hlásí pomocí výjimek. Jak na ně bude reagovat
uživatel-programátor při návrhu svého programu, je čistě na jeho volbě.

\end{document}
